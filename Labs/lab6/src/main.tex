%%%%%%%%%%%%%%%%%%%%%%%%%%%%%%%%%%%%%%%%%
% Short Sectioned Assignment
% LaTeX Template
% Version 1.0 (5/5/12)
%
% This template has been downloaded from: http://www.LaTeXTemplates.com
% Original author: % Frits Wenneker (http://www.howtotex.com)
% License: CC BY-NC-SA 3.0 (http://creativecommons.org/licenses/by-nc-sa/3.0/)
% Modified by Alan G. Labouseur  - alan@labouseur.com, and Ryan Munger - ryan.munger1@marist.edu
%
%%%%%%%%%%%%%%%%%%%%%%%%%%%%%%%%%%%%%%%%%

%----------------------------------------------------------------------------------------
%	PACKAGES AND OTHER DOCUMENT CONFIGURATIONS
%----------------------------------------------------------------------------------------

\documentclass[letterpaper, 10pt]{article} 

\usepackage{tikz}
\usetikzlibrary{automata, positioning}
\usepackage[english]{babel} % English language/hyphenation
\usepackage{graphicx}
\usepackage[lined,linesnumbered,commentsnumbered]{algorithm2e}
\usepackage{listings}
\usepackage{float}
\usepackage{fancyhdr} % Custom headers and footers
\pagestyle{fancyplain} % Makes all pages in the document conform to the custom headers and footers
\usepackage{lastpage}
\usepackage{url}
\usepackage{xcolor}
\usepackage{titlesec}
\usepackage{ulem}

% Stolen from https://www.overleaf.com/learn/latex/Code_listing 
\definecolor{codegreen}{rgb}{0,0.6,0}
\definecolor{codegray}{rgb}{0.5,0.5,0.5}
\definecolor{codepurple}{rgb}{0.58,0,0.82}
\definecolor{backcolour}{rgb}{0.95,0.95,0.92}

\lstdefinestyle{mystyle}{
    backgroundcolor=\color{backcolour},   
    commentstyle=\color{codegreen},
    keywordstyle=\color{magenta},
    numberstyle=\tiny\color{codegray},
    stringstyle=\color{codepurple},
    basicstyle=\ttfamily\footnotesize,
    breakatwhitespace=false,         
    breaklines=true,                 
    captionpos=b,                    
    keepspaces=true,                 
    numbers=left,                    
    numbersep=5pt,                  
    showspaces=false,                
    showstringspaces=false,
    showtabs=false,                  
    tabsize=2
}
\lstset{style=mystyle, language=c++}


\fancyhead{} % No page header - if you want one, create it in the same way as the footers below
\fancyfoot[L]{} % Empty left footer
\fancyfoot[C]{page \thepage\ of \pageref{LastPage}} % Page numbering for center footer
\fancyfoot[R]{}

\renewcommand{\headrulewidth}{0pt} % Remove header underlines
\renewcommand{\footrulewidth}{0pt} % Remove footer underlines
\setlength{\headheight}{13.6pt} % Customize the height of the header

%----------------------------------------------------------------------------------------
%	TITLE SECTION
%----------------------------------------------------------------------------------------

\newcommand{\horrule}[1]{\rule{\linewidth}{#1}} % Create horizontal rule command with 1 argument of height

\title{	
   \normalfont \normalsize 
   \textsc{CMPT 432 - Spring 2025 - Dr. Labouseur} \\[10pt] % Header stuff.
   \horrule{0.5pt} \\[0.25cm] 	% Top horizontal rule
   \huge Lab 6 -- Symbol Tables \\     	    % Assignment title
   \horrule{0.5pt} \\[0.25cm] 	% Bottom horizontal rule
}

\author{Ryan Munger \\ \normalsize Ryan.Munger1@marist.edu}

\date{\normalsize\today} 	% Today's date.

\begin{document}

\maketitle % Print the title

%----------------------------------------------------------------------------------------
%   CONTENT SECTION
%----------------------------------------------------------------------------------------

% - -- -  - -- -  - -- -  -
\vspace{-.7cm}
\section{Crafting a Compiler}
\subsection{8.1 -- Binary Search Trees \& Hash Tables}
The two data structures most commonly used to implement symbol
tables in production compilers are binary search trees and hash tables.
What are the advantages and disadvantages of using each of these data
structures for symbol tables? \\

\noindent
\textbf{Binary Search Tree} \\
Advantages:
\begin{enumerate}
    \item Simple \& widely known implementation
    \item Average insertion/lookup is $O(log(n))$
    \item No need to compute hash; just compare
\end{enumerate}
Disadvantages:
\begin{enumerate}
    \item Symbol names are not random - may degrade performance to $O(n)$
    \item May require complex/expensive algorithm to balance the tree
\end{enumerate} 

\newpage
\noindent
\textbf{Hash Table} \\
Advantages:
\begin{enumerate}
    \item Constant time lookups $O(1 + \alpha)$
    \item Great for symbol table's high need for lookups
\end{enumerate}
Disadvantages:
\begin{enumerate}
    \item Need to compute hash
    \item Do not store symbol names in any sorted order if needed
\end{enumerate} 

\vspace{.5cm}
\noindent
I'm on team hash table! 

\section{Dragon Book}
No exercises this time, just absorb the book!!


\end{document}